\begin{frame}
	\frametitle{Comentários}

	\begin{figure}[h]
		\centering
			\includegraphics[height=0.6\paperheight]{figuras/johann}
		\caption{Retirada do site \href{https://www.pensador.com/frase/MzI2/}{Pensador}}\label{figure:johann}
	\end{figure}

\end{frame}

\begin{frame}
	\frametitle{Problemas}

	\begin{itemize}
		\item Comentários tendem a mentir
		\item Você não sabe pra que serve um código comentado
		\item Comentários tentam explicar o que não está claro em seu código
		\item * Comentários podem ser lixos no código
	\end{itemize}

\end{frame}

\begin{frame}
	\Huge Exemplo
\end{frame}

\begin{frame}[fragile]
	\frametitle{O que essa função deve fazer?}

	\begin{listing}[H]
		\caption{Comentário para explicar o que a função faz}
		\begin{minted}[baselinestretch=1.2,fontsize=\scriptsize,linenos]{c}
/*
	Função para contar o tamanho da string
	char *string String a ter seu tamanho contado
	Retorno tamanho da string
*/
int contarTamanho(char *string);
		\end{minted}

	\end{listing}

\end{frame}

\begin{frame}[fragile]
	\frametitle{O que a função faz de fato?}

	\begin{listing}[H]
		\caption{Comentário para explicar o que a função faz}
		\begin{minted}[baselinestretch=1.0,fontsize=\scriptsize,linenos]{c}
/*
	Funcao para contar o tamanho da string
	char *string String a ter seu tamanho contado
	Retorno tamanho da string
*/
int contarTamanho(char *string){
	int tamanho=0;
	int i=0;

	while(string[i]!='\0' && i < strlen(string)){
		if(string[i]==' '){
			string[i]='\0';
		}
		tamanho++;
		i++;
	}

	return tamanho;
}
		\end{minted}
	\end{listing}

\end{frame}

\begin{frame}
	\Huge Caso real
\end{frame}

\begin{frame}[fragile]

	\begin{listing}[H]
		\caption{Comentários em excesso}
		\begin{minted}[baselinestretch=1.0,fontsize=\scriptsize,linenos]{java}
/**
* Metodo construtor de Ocorrencia.
*
* @param dataHora Data e hora da ocorrência.
* @param transitavelVeiculo Transitável por meio de Veiculo.
* @param transitavelAPe Transitável a pé.
* @param descricao Descrição da ocorrência.
* @param qtdExistente Quantidade de existente.
* @param qtdInexistente Quantidade de inexistente.
* @param qtdCasoEncerrado Quantidade de caso encerrado.
* @param pkOcorrencia Número de identificação da ocorrência
* @param tipo Tipo da ocorrência.
* @param user Objeto Usuário.
*/
public Ocorrencia(Date dataHora, boolean transitavelVeiculo, boolean transitavelAPe,
			String descricao, long qtdExistente, long qtdInexistente,
			long qtdCasoEncerrado, long pkOcorrencia, Tipo tipo, Usuario user) {
	this.dataHora = (Date) dataHora.clone();
 	 //...
}
		\end{minted}
	\end{listing}

\end{frame}

\begin{frame}
	\Huge Exemplo
\end{frame}

\begin{frame}[fragile]

	\begin{listing}[H]
		\caption{Comentário para explicar cada trecho de código}
		\begin{minted}[baselinestretch=1.2,fontsize=\scriptsize,linenos]{c}
/*
	Função para calcular a área de um retângulo
	float b um lado do retângulo
	float c outro lado do retângulo
	Retorno área do retângulo
*/
float calc (float b, float c){
	float a = b * c; // a recebe b * c
	return a; // retorna a
}
		\end{minted}
	\end{listing}

\end{frame}

\begin{frame}
	\Huge Exemplo
\end{frame}

\begin{frame}[fragile]
	\frametitle{O que fazer com código comentado?}

	\begin{listing}[H]
		\caption{Código comentado}
		\begin{minted}[baselinestretch=1.2,fontsize=\scriptsize,linenos]{verilog}
module inicial ( giro, entrada, saida, metais, ledVerde, ledVermelho, display, clock );
	input giro, entrada, saida, metais, clock;
	output [1:0] ledVerde, ledVermelho;
	output [6:0] display;

	reg [3:0] tmp;
	[...]
	tmp = { giro, entrada, saida, metais };
	/*
		tmp[3] = giro;
		tmp[2] = entrada;
		tmp[1] = saida;
		tmp[0] = metais;
	*/

	[...]
endmodule
		\end{minted}
	\end{listing}

\end{frame}

\begin{frame}[fragile]

	\begin{listing}[H]
		\caption{Código comentado}
		\begin{minted}[baselinestretch=1.2,fontsize=\scriptsize,linenos]{verilog}
module inicial ( giro, entrada, saida, metais, ledVerde, ledVermelho, display, clock );
	input giro, entrada, saida, metais, clock;
	output [1:0] ledVerde, ledVermelho;
	output [6:0] display;

	reg [3:0] tmp;
	[...]
	tmp = { giro, entrada, saida, metais };// Equivalente ao código comentado
	/*
		tmp[3] = giro;
		tmp[2] = entrada;
		tmp[1] = saida;
		tmp[0] = metais;
	*/

	[...]
endmodule
		\end{minted}
	\end{listing}

\end{frame}
