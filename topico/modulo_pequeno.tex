\begin{frame}
	\frametitle{Módulos grandes (não especializados)}

	\begin{figure}[h]
		\centering
			\includegraphics[height=0.6\paperheight]{figuras/george}
		\caption{Retirada do site \href{https://www.pensador.com/frase/OTMw/}{Pensador}}\label{figure:george}
	\end{figure}

\end{frame}

\begin{frame}
	\frametitle{Problemas}

	\begin{itemize}
		\item Fazer diversas coisas (resolver diversos problemas)
		\item Alto grau de complexidade
		\item Baixa reutilização
		\item Alto custo de manutenção
		\item Podem não ser tão eficientes
		\item Podem esconder problemas
		\item Modificar um trecho que não deveria
	\end{itemize}
\end{frame}

\begin{frame}
	\Huge Caso real
\end{frame}

\begin{frame}[fragile]
	\frametitle{Onde está o problema?}

	\begin{listing}[H]
		\begin{minted}[baselinestretch=1.2,fontsize=\scriptsize,linenos]{c}
[...]
FILE *pArquivo;
int nLinhas = 0;
char caracter;
char palavra[100];

if((pArquivo=fopen("meuArquivo.txt", "r")) != NULL){
	while(fscanf(pArquivo, "%c", &caracter) != EOF){
		if(caracter == '\n'){
		  nLinhas+=1;
		}
	}
	for(i=0; i<rand()%nLinhas+1; i++){
		fscanf(pArquivo, "%s\n", &palavra);
	}
fclose(pArquivo);
[...]
		\end{minted}
	\end{listing}

\end{frame}

\begin{frame}[fragile]
	\frametitle{Onde está o problema?}

	\begin{listing}[H]
		\begin{minted}[baselinestretch=1.0,fontsize=\scriptsize,linenos]{c}
[...]
FILE *pArquivo;
int nLinhas = 0;
char caracter;
char palavra[100];

if((pArquivo=fopen("meuArquivo.txt", "r")) != NULL){
	while(fscanf(pArquivo, "%c", &caracter) != EOF){
		if(caracter == '\n'){
		  nLinhas+=1;
		}
	}
	//Aqui está o problema, pois o ponteiro pArquivo não está mais na posição inicial
	for(i=0; i<rand()%nLinhas+1; i++){
		fscanf(pArquivo, "%s\n", &palavra);
	}
fclose(pArquivo);
[...]
		\end{minted}
	\end{listing}

\end{frame}

\begin{frame}[fragile]
	\frametitle{Solução porca}

	\begin{listing}[H]
		\begin{minted}[baselinestretch=1.0,fontsize=\scriptsize,linenos]{c}
[...]
FILE *pArquivo;
int nLinhas = 0;
char caracter;
char palavra[100];

if((pArquivo=fopen("meuArquivo.txt", "r")) != NULL){
	while(fscanf(pArquivo, "%c", &caracter) != EOF){
		if(caracter == '\n'){
			nLinhas+=1;
		}
	}

	rewind(pArquivo);
	for(i=0; i<rand()%nLinhas+1; i++){
		fscanf(pArquivo, "%s\n", &palavra);
	}
fclose(pArquivo);
[...]
		\end{minted}
	\end{listing}

\end{frame}

\begin{frame}[fragile]
	\frametitle{Solução ideal}

	\begin{listing}[H]
		\begin{minted}[baselinestretch=1.0,fontsize=\scriptsize,linenos]{c}
void lerLinhaSorteada(char *nomeDoArquivo){
	int quantidadeDeLinhas = contarNumeroDeLinhas(nomeDoArquivo);

	if(quantidadeDeLinhas == -1)
		return;

	int linha = sortear(quantidadeDeLinhas);

	FILE *arquivo = fopen(nomeDoArquivo, "r");

	moverPonteiroArquivo(&arquivo, linha);

	char palavra[100];
	fscanf(arquivo,"%s\n",palavra);
	fclose(arquivo);

	printf("%s\n",palavra);
}
		\end{minted}
	\end{listing}

\end{frame}

\begin{frame}[fragile]
	\frametitle{Solução ideal}

	\begin{listing}[H]
		\begin{minted}[baselinestretch=1.0,fontsize=\scriptsize,linenos]{c}
int contarNumeroDeLinhas(char *nomeDoArquivo){
    int quantidadeDeLinhas = 0;
    char caracter;
    FILE *arquivo = fopen(nomeDoArquivo, "r");

    if(arquivo ==NULL){
        quantidadeDeLinhas = -1;
    }else{
        while(fscanf(arquivo, "%c", &caracter) != EOF){
            if(caracter == '\n'){
                quantidadeDeLinhas++;
            }
        }

        fclose(arquivo);
    }

    return quantidadeDeLinhas;
}
		\end{minted}
	\end{listing}

\end{frame}

\begin{frame}[fragile]
	\frametitle{Solução ideal}

	\begin{listing}[H]
		\begin{minted}[baselinestretch=1.2,fontsize=\scriptsize,linenos]{c}
int sortear(int limite){
	return rand()%limite+1;
}
		\end{minted}
	\end{listing}

\end{frame}

\begin{frame}[fragile]
	\frametitle{Solução ideal}

	\begin{listing}[H]
		\begin{minted}[baselinestretch=1.2,fontsize=\scriptsize,linenos]{c}
void moverPonteiroArquivo(FILE **arquivo, int linha){
    char palavra[100];

    for(int i=0; i<linha-1; i++){
        fscanf(*arquivo, "%s\n",&palavra);
    }
}
		\end{minted}
	\end{listing}

\end{frame}

\begin{frame}
	\frametitle{Solução}

	\begin{itemize}
		\item Criar funções especializadas
	\end{itemize}
\end{frame}
